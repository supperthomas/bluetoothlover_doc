%% Generated by Sphinx.
\def\sphinxdocclass{report}
\documentclass[letterpaper,10pt,english]{sphinxmanual}
\ifdefined\pdfpxdimen
   \let\sphinxpxdimen\pdfpxdimen\else\newdimen\sphinxpxdimen
\fi \sphinxpxdimen=.75bp\relax
%% turn off hyperref patch of \index as sphinx.xdy xindy module takes care of
%% suitable \hyperpage mark-up, working around hyperref-xindy incompatibility
\PassOptionsToPackage{hyperindex=false}{hyperref}

\PassOptionsToPackage{warn}{textcomp}

\catcode`^^^^00a0\active\protected\def^^^^00a0{\leavevmode\nobreak\ }
\usepackage{cmap}
\usepackage{xeCJK}
\usepackage{amsmath,amssymb,amstext}
\usepackage{babel}



\setmainfont{FreeSerif}[
  Extension      = .otf,
  UprightFont    = *,
  ItalicFont     = *Italic,
  BoldFont       = *Bold,
  BoldItalicFont = *BoldItalic
]
\setsansfont{FreeSans}[
  Extension      = .otf,
  UprightFont    = *,
  ItalicFont     = *Oblique,
  BoldFont       = *Bold,
  BoldItalicFont = *BoldOblique,
]
\setmonofont{FreeMono}[
  Extension      = .otf,
  UprightFont    = *,
  ItalicFont     = *Oblique,
  BoldFont       = *Bold,
  BoldItalicFont = *BoldOblique,
]


\usepackage[Sonny]{fncychap}
\ChNameVar{\Large\normalfont\sffamily}
\ChTitleVar{\Large\normalfont\sffamily}
\usepackage{sphinx}

\fvset{fontsize=\small}
\usepackage{geometry}


% Include hyperref last.
\usepackage{hyperref}
% Fix anchor placement for figures with captions.
\usepackage{hypcap}% it must be loaded after hyperref.
% Set up styles of URL: it should be placed after hyperref.
\urlstyle{same}

\addto\captionsenglish{\renewcommand{\contentsname}{BLE knowledge:}}

\usepackage{sphinxmessages}
\setcounter{tocdepth}{1}



\title{supperthomas\_wiki}
\date{2020 年 09 月 26 日}
\release{0.0.1}
\author{supperthomas}
\newcommand{\sphinxlogo}{\vbox{}}
\renewcommand{\releasename}{发布}
\makeindex
\begin{document}

\ifdefined\shorthandoff
  \ifnum\catcode`\=\string=\active\shorthandoff{=}\fi
  \ifnum\catcode`\"=\active\shorthandoff{"}\fi
\fi

\pagestyle{empty}
\sphinxmaketitle
\pagestyle{plain}
\sphinxtableofcontents
\pagestyle{normal}
\phantomsection\label{\detokenize{index::doc}}



\chapter{BLE 广播\sphinxhyphen{}(1)}
\label{\detokenize{test/ble_adv_scan_all:ble-1}}\label{\detokenize{test/ble_adv_scan_all::doc}}

\section{1. 基本知识介绍}
\label{\detokenize{test/ble_adv_scan_all:id1}}
查看本篇文章,大家可以知道以下内容:
\begin{itemize}
\item {} 
ble是如何打广播的,让对方发现自己的。

\item {} 
广播数据如何修改

\end{itemize}


\subsection{1.1 基本概念}
\label{\detokenize{test/ble_adv_scan_all:id2}}
BLE的物理层这边简单提一些。

\sphinxincludegraphics{{adv_channel}.png}

ble的信道和BR/EDR的信道是完全不一样的。但是范围是相同的,差不多也都是2.4Ghz的频道。可以简单理解为空中有40个信道0\textasciitilde{}39信道。两个设备在相同的信道里面可以进行相互通信。

而这些信道SIG又重新编号:

\sphinxincludegraphics{{adv_channel2}.png}

这个编号就是把37 38 39。 3个信道抽出来,作为广播信道,其他都是数据信道。这篇文章主要讲广播,所以基本数据信息都是围绕37 38 39这三个信道上面的通信来讲的。

我们可以看到这3个信道是分散排列的。大家可以思考下为什么。

其实看下面一张图就知道了。

\sphinxincludegraphics{{wifi_channel}.png}


\subsection{1.2 core spec的内容}
\label{\detokenize{test/ble_adv_scan_all:core-spec}}
本文所讲的内容主要在

BLUETOOTH CORE SPECIFICATION Version 5.2 | Vol 6, Part B

2.3 ADVERTISING PHYSICAL CHANNEL PDU

章节中,想要仔细研究的可以研究该章节,本文会将比较常用的内容展示给大家。

GAP中也有部分定义广播相关的内容

BLUETOOTH CORE SPECIFICATION Version 5.2 | Vol 3, Part C

9 OPERATIONAL MODES AND PROCEDURES – LE
PHYSICAL TRANSPORT


\subsection{1.3 基本理解}
\label{\detokenize{test/ble_adv_scan_all:id3}}
我们使用手机蓝牙通常会搜索设备。搜到的设备有两种,ble和BR/EDR的设备,BR/EDR是和 经典蓝牙相关的,本文今天不介绍,下面来介绍ble相关的操作。其实仔细思考一下手机搜索ble设备的时候,手机其实就充当一个观察者(observer),ble设备其实就是一个广播者(broadcaster)。

大家可以用手机下载一个apk应用,nrf connect.apk。苹果手机,可以使用lightblue

!{[}image\sphinxhyphen{}20200809080327101{]}(picture/nrf connect.jpg)


\section{2. 广播内容(adv data)}
\label{\detokenize{test/ble_adv_scan_all:adv-data}}
我们先来理解一下最基本的广播ADV\_IND

\sphinxincludegraphics{{ADV_IND}.png}

这张图的大概意思是:ADV\_IND广播有两部分组成,1. 广播地址(就是广播者的地址,占6个字节)2. 广播数据(0\sphinxhyphen{}31个字节)(今天讲的主要内容一共就31个字节,听起来是不是很简单)

从上面nrf connect的软件中可以看到蓝牙地址,这个地址就是广播的地址,其他的所有你能看到的内容都在后面的31个字节里面。如果能理解这31个字节的内容,基本上你就可以熟练的使用蓝牙广播功能了。

这31个字节里面内容,

\sphinxincludegraphics{{adv_data}.png}

\sphinxincludegraphics{{adv_data2}.png}

图中的data就是31个byte。

这个adv\_data中都是由一个一个的小元素组成的。称之为AD Structure。

每个元素里面有两个要素:1. 长度(length), 2. 数据(data)

每个数据里面又包含两个元素:1. 类型(type), 2. 数据

总结一下就是一个L T V模型(length, type, data)

这个length代表的是后面数据有多长,不包含length的长度。

我们拿一个常见的广播数据来讲一下即可。

这个广播数据可以用nrf connect 来获取。

\sphinxincludegraphics{{adv1}.png}

点击RAW,可以看到数据:

\sphinxincludegraphics{{adv2}.png}

\begin{sphinxVerbatim}[commandchars=\\\{\}]
\PYG{l+m+mh}{0x02} \PYG{l+m+mh}{0x01} \PYG{l+m+mh}{0x1a} 
\PYG{l+m+mh}{0x04} \PYG{l+m+mh}{0x09} \PYG{l+m+mh}{0x62} \PYG{l+m+mh}{0x6c} \PYG{l+m+mh}{0x65}
\PYG{l+m+mh}{0x02} \PYG{l+m+mh}{0x0a} \PYG{l+m+mh}{0xf9}
\end{sphinxVerbatim}

这里面一共放了3个信息。

第一个byte是长度,第二个byte是ad type。

这边可能你就要像知道ad type到底是什么意思呢?

其实软件已经帮您解释出来了。而且core spec里面其实也没有ad type是什么意思的完全解释。

记住这点,core spec里面没有解释。

那哪里有呢?其实细心一点你可以发现:

\sphinxincludegraphics{{adv_type}.png}

这个网址比较旧了。https://www.bluetooth.com/specifications/assigned\sphinxhyphen{}numbers/generic\sphinxhyphen{}access\sphinxhyphen{}profile/

可以访问这个网址。

\sphinxincludegraphics{{ad_type}.png}

我们这里看到了0x01代表的是flags。而这个flags里面值代表什么含义呢?

后面提供了索引信息(但是这个索引信息有一些旧了,建议大家不用参考)。

主要参考一份文档

这个文档中有所有AD TYPE的类型描述,下面我就简单讲下上面所提到的3个常用的AD TYPE
\begin{itemize}
\item {} 
0x01 FLAGS

这个是标志该设备是哪一种类型的,有LE 和BR/EDR NOT SUPPORT是常见的,其他的不太常用,这个值也不太常需要改变

\end{itemize}

\sphinxincludegraphics{{FLAGS}.png}
\begin{itemize}
\item {} 
0x09  complelte local name

\sphinxincludegraphics{{ad_type2}.png}

\end{itemize}

这个现实的是名称,就是你要给手机显示的名字,后面3个byte ascii就是“ble”

所以app上面会显示该名称,这个也不用查手册,后面就是具体的名称,长度在第一个字节0x04里面有体现
\begin{itemize}
\item {} 
0x0a  tx power level

\sphinxincludegraphics{{tx_power}.png}

\end{itemize}

!{[}image\sphinxhyphen{}20200809100844900{]}(./picture/tx power2.png)

这里面就很明显了,上面那个值是0xF9 代表的是\sphinxhyphen{}7dm(这个是补码显示的)

在app上面也有体现:

\sphinxincludegraphics{{txpower2}.png}

熟悉了上面的内容,基本就可以知道广播内容是如何显示,以及31个字节是如何写的了,这个相当于是应用层,接下来,会深入介绍协议栈层是如何设置之类的。


\section{3. 广播参数和HCI 命令(ADV)}
\label{\detokenize{test/ble_adv_scan_all:hci-adv}}
​          上面讲到了广播内容的设定,但是这些内容要怎么展现呢?这就需要了解本章节了。


\subsection{3.1 adv enable}
\label{\detokenize{test/ble_adv_scan_all:adv-enable}}
​        先讲一下这个是能命令,有了这个命令就可以开始打广播了

!{[}image\sphinxhyphen{}20200809151314184{]}(./picture/adv enable.png)

这个命令有一个参数,

!{[}image\sphinxhyphen{}20200809151339362{]}(./picture/adv enable param.png)

有了这个命令你就可以打广播了,广播内容可能为空也可能是默认值,不管如何,总之有了这一条命令你就可以控制是否开始打广播了。

\sphinxincludegraphics{{airlog}.png}

\begin{sphinxVerbatim}[commandchars=\\\{\}]
\PYG{l+m+mi}{01} \PYG{l+m+mi}{0}\PYG{n}{a} \PYG{l+m+mi}{20} \PYG{l+m+mi}{01} \PYG{l+m+mi}{01}
\end{sphinxVerbatim}

通过HCI 给卡片发这条命令,就可以用nrf connect扫描看到设备了。


\subsection{3.2 set adv data}
\label{\detokenize{test/ble_adv_scan_all:set-adv-data}}
​       设置广播数据内容

!{[}image\sphinxhyphen{}20200809152243790{]}(./picture/set adv data.png)

参数:

!{[}image\sphinxhyphen{}20200809152316460{]}(./picture/set adv param .png)

这个也很好理解,就是上一章讲到的广播内容(adv data),这个就是设置广播内容的命令。参数就是内容的长度和内容的数据,最大也就31个字节。


\subsection{3.3 set adv param}
\label{\detokenize{test/ble_adv_scan_all:set-adv-param}}
这个广播参数就相对来说比较复杂一些了,不过可以先留个印象,后面看空气包的时候可以结合一起来看。

命令:

!{[}image\sphinxhyphen{}20200809152750407{]}(./picture/adv param.png)

参数:

\sphinxincludegraphics{{image-20200809152830009}.png}

!{[}image\sphinxhyphen{}20200809153236345{]}(./picture/adv param2.png)

这个实在是太多了,我就不一一讲了,记住这边会有一个很多的参数,后面讲空气包的时候会联系到这边一起讲。

最主要的理解这3个就可以了,可能还有一些其他连带的命令,比如tx power之类的,其他的都是协议栈内容。


\section{5. 空中传输}
\label{\detokenize{test/ble_adv_scan_all:id4}}
!{[}image\sphinxhyphen{}20200809163421555{]}(./picture/air log.png)

当发送命令adv enable的时候,蓝牙卡片就会在3个通道(37, 38,39)上发送数据,其实就是在3个通道上面发送数据,发送的数据就是这个广播什么类型,以及广播内容(就是上面讲到的adv data)

在空气中,这个包空气中如何发送,以及以多大的间隔发送,都是根据set adv param。


\subsection{5.2 set adv param举例:}
\label{\detokenize{test/ble_adv_scan_all:id5}}
我们来举个例子看看

\sphinxincludegraphics{{adv_param}.png}

\begin{sphinxVerbatim}[commandchars=\\\{\}]
\PYG{n}{adv} \PYG{n}{intreval} \PYG{n+nb}{min}\PYG{p}{:} \PYG{l+m+mi}{30}\PYG{n}{ms}\PYG{p}{(}\PYG{l+m+mi}{48} \PYG{n}{slot}\PYG{p}{)}
\PYG{n}{adv} \PYG{n}{interval} \PYG{n+nb}{max}\PYG{p}{:} \PYG{l+m+mi}{60}\PYG{n}{ms}\PYG{p}{(}\PYG{l+m+mi}{96} \PYG{n}{slot}\PYG{p}{)}
\PYG{n}{adv} \PYG{n+nb}{type} \PYG{p}{:} \PYG{n}{connectable} \PYG{n}{undirected} \PYG{n}{advertising}
\PYG{n}{own} \PYG{n}{address} \PYG{n+nb}{type}\PYG{p}{:} \PYG{n}{public}
\PYG{n}{direct} \PYG{n}{address} \PYG{n+nb}{type}\PYG{p}{:} \PYG{n}{public}
\PYG{n}{direct} \PYG{n}{address}\PYG{p}{:} \PYG{l+m+mi}{00}\PYG{p}{:}\PYG{l+m+mi}{00}\PYG{p}{:}\PYG{l+m+mi}{00}\PYG{p}{:}\PYG{l+m+mi}{00}\PYG{p}{:}\PYG{l+m+mi}{00}\PYG{p}{:}\PYG{l+m+mi}{00}
\PYG{n}{adv} \PYG{n}{channel} \PYG{n+nb}{map}\PYG{p}{:} \PYG{l+m+mi}{37} \PYG{l+m+mi}{38} \PYG{l+m+mi}{39} \PYG{n}{enable}
\PYG{n}{adv} \PYG{n+nb}{filter} \PYG{n}{policy}\PYG{p}{:} \PYG{n}{scan} \PYG{n}{request} \PYG{k+kn}{from} \PYG{n+nn}{any}\PYG{p}{,}\PYG{n}{connect} \PYG{n}{request} \PYG{k+kn}{from} \PYG{n+nn}{any}
\end{sphinxVerbatim}

看下时间间隔

\sphinxincludegraphics{{adv_interval}.png}

简单理解一下:

\begin{sphinxVerbatim}[commandchars=\\\{\}]
1. adv interval 这个是指示的每次发送广播的时间间隔,这边设置的是30\PYGZti{}60ms,从空中看有一次是33ms,就是发送广播的时间间隔,看上去也是符合的。
2. adv type: 这个是广播类型,不同的广播类型,发送的广播类型不一样,主机认识的类型也不一样,下一个章节会介绍一些常见的类型,
3. own address type: 本机的蓝牙地址类型,public的,这个涉及到蓝牙地址类型的知识,不展开了。
4. direct adress: 发送direct广播的时候需要参考这个地址,也分类型和地址
5. adv channel: 这个就是37 38 39 需要在哪几个通道里面,基本上默认都是全通道打的。
6. adv filter policy: 这个就是响应哪些请求。
\end{sphinxVerbatim}

通常我们常用的就是adv interval,这个可以控制打广播的时间间隔,具体每次打的时间间隔底层根据范围随机来设置的。控制这个时间间隔可以降低打广播时候的功耗。


\section{6. 广播的种类}
\label{\detokenize{test/ble_adv_scan_all:id6}}
介绍几种常用的广播类型,这个涉及到的参数就是adv param中的adv type


\subsection{6.1 ADV\_IND}
\label{\detokenize{test/ble_adv_scan_all:adv-ind}}
​    这个比较常用的,可以连接的,非定向的,任何设备都可以搜到的。


\subsection{6.2 ADV\_DIRECT\_IND}
\label{\detokenize{test/ble_adv_scan_all:adv-direct-ind}}
​    定向广播,这个就只能特定设备才能搜到该广播,也就是地址是direct address的设备才能搜到该广播,(实际上空气中还是会有的,只是对端HOST不会上报)


\subsection{6.3 ADV\_NONCONN\_IND}
\label{\detokenize{test/ble_adv_scan_all:adv-nonconn-ind}}
​     非定向不可连接的广播,这个就是告诉对端,该设备不可连接,在nrf connect上面也会看到设备是没有connect按钮的。


\subsection{6.4 SCAN\_REQ}
\label{\detokenize{test/ble_adv_scan_all:scan-req}}
​        这个也是一种广播,不过这种广播是请求对端的scan response data, scan response 数据可以理解成广播数据的补充。可以不同,也可以相同。


\subsection{6.5 SCAN\_RESPONSE}
\label{\detokenize{test/ble_adv_scan_all:scan-response}}
​         这个和6.4相对应,作为回应,回复的也是scan response的数据。有些重要的数据可以放到scan req中,非重要的数据可以放到sccan response中。


\section{7. beacon}
\label{\detokenize{test/ble_adv_scan_all:beacon}}
其实对于协议栈来说,发什么广播数据都不用关心。31个byte怎么传输都可以,所以这个beacon相对来说,当需要开发相应的应用的时候需要非常了解,不开发的话也可以不用了解,这个我确实也不是特别了解里面的内容,大部分自成体系的。我就摘抄网上讲的比较好的记录一下:

​       目前主流的三种帧格式分别为苹果公司的iBeacon,Radius Networks公司的AltBeacon以及谷歌公司的Eddystone。


\subsection{7.1 Ibeacon}
\label{\detokenize{test/ble_adv_scan_all:ibeacon}}
iBeacon使用了称为厂商数据字段的标准AD Type结构。如下图所示,为iBeacon的广播包,按AD Type结构进行分割如下:

\sphinxincludegraphics{{ibeacon}.png}

\sphinxincludegraphics{{ibeacon2}.png}

厂商数据字段的数据域前2字节为公司识别码。由蓝牙SIG组织分配给各公司,指示后续数据的解码方式。在上图中,0x004C为苹果公司的ID。0x02指明该设备为“proximity  beacon”,该值在iBeacon设备中均为0x02。UUID指明拥有该beacon设备的机构。主次字段用来编码位置信息,通常主字段指明某个建筑,而次字段指明在这栋建筑中的特定位置。例如“伦敦中心商场,运动产品区”。发送功率字段帮助应用进行距离估算。有关iBeacon的详细内容可以参考\sphinxhref{https://developer.apple.com/ibeacon/Getting-Started-with-iBeacon.pdf}{Getting started with iBeacon}


\subsection{7.2 Altbeacon}
\label{\detokenize{test/ble_adv_scan_all:altbeacon}}
\sphinxincludegraphics{{altbeacon}.png}


\subsection{7.3 eddystone}
\label{\detokenize{test/ble_adv_scan_all:eddystone}}
谷歌公司的Eddystone与iBeacon及AltBeacon有所不同。它没用使用所谓的厂商数据字段,而是使用16位服务UUID字段以及服务数据字段。Eddystone还定义了如下图所示的子类型,具体内容可以参考\sphinxhref{https://github.com/google/eddystone}{eddystone}:

\sphinxincludegraphics{{eddyston}.png}

这边提供一个blog供需要的人参考吧。

https://blog.csdn.net/bi\_jian/article/details/82927904

可以这样理解,其实beacon的一些应用不太需要协议栈的一些链路内容,只要可以打广播即可。


\chapter{BLE 扫描(2)}
\label{\detokenize{test/ble_adv_scan_all:ble-2}}

\section{8. 主机scan扫描}
\label{\detokenize{test/ble_adv_scan_all:scan}}
通常扫描是作为主机端来控制的,就是我们通常的手机端,所以这部分放到最后,可能只做丛机有不需要了解下面的内容。


\subsection{8.1 scan相关的HCI 命令和event}
\label{\detokenize{test/ble_adv_scan_all:scanhci-event}}
跟上面广播命令相对应。


\subsubsection{8.1.1 scan enable}
\label{\detokenize{test/ble_adv_scan_all:scan-enable}}
这个命令就是扫描的开关了,打开扫描,

!{[}image\sphinxhyphen{}20200810211520195{]}(./picture/scan enable.png)

两个参数:
\begin{itemize}
\item {} 
scan enable

!{[}image\sphinxhyphen{}20200810211603006{]}(./picture/scan enable2.png)

\item {} 
filter\_dumplicate

这个参数就是是否过滤重复的信息,默认是打开的,如果不打开,scan到一次就会上报一次,不过滤重复地址。

\sphinxincludegraphics{{filter_dumplicate}.png}

\end{itemize}


\subsubsection{8.1.2 set scan data}
\label{\detokenize{test/ble_adv_scan_all:set-scan-data}}
这个和上面的adv data内容差不多

!{[}image\sphinxhyphen{}20200810211924445{]}(./picture/scan data.png)


\subsubsection{8.1.3 set scan param}
\label{\detokenize{test/ble_adv_scan_all:set-scan-param}}
参数和adv param类似:

!{[}image\sphinxhyphen{}20200810212145243{]}(./picture/scan param.png)


\subsubsection{8.1.4 EVENT  LE adv report}
\label{\detokenize{test/ble_adv_scan_all:event-le-adv-report}}
做扫描,扫描到设备的时候,会上报该条event

!{[}image\sphinxhyphen{}20200810213335491{]}(./picture/adv report event.png)

这个里面有以下参数:

!{[}image\sphinxhyphen{}20200810213555256{]}(./picture/repo param.png)


\subsection{8.2  SCAN 类型}
\label{\detokenize{test/ble_adv_scan_all:id7}}
scan分为Passive scan和Active scan


\subsubsection{8.2.1 被动扫描(passive scan)}
\label{\detokenize{test/ble_adv_scan_all:passive-scan}}
!{[}image\sphinxhyphen{}20200810212526908{]}(./picture/passive scan.png)

被动扫描,这个主要看上面这个流程,对方发广播了,扫描到了,就回报给host。


\subsubsection{8.2.2 主动扫描(Active scan)}
\label{\detokenize{test/ble_adv_scan_all:active-scan}}
!{[}image\sphinxhyphen{}20200810212942242{]}(./picture/active scan.png)

这个主动扫描,就是开启扫描之后,如果搜到了广播,发送SCAN\_REQ请求,之后搜到SCAN\_RSP之后再上报信息。


\subsection{8.3 SCAN 参数}
\label{\detokenize{test/ble_adv_scan_all:id8}}
scan在空气中实际上是不太能看到的,因为主机处于被动接收,所以空气包中也看不到scan的参数,

下面画一张图让大家理解scan window和scan interval的意思。主机会在同一个频道内,听一个scan window,一个scan interval换一次频率。

!{[}image\sphinxhyphen{}20200810221251125{]}(./picture/SCAN I.png)

常用的就是这个interval和window,居然时间可以参考上面的手册。


\subsection{8.4 mesh}
\label{\detokenize{test/ble_adv_scan_all:mesh}}
其实理解了上面的大部分内容,基本上mesh可以做adv相关的承载了。其实mesh有了上面的知识,基本上就可以跑了。其他的都可以暂时不用看。


\subsection{8.4 ble 5.0 \& ble 4.0}
\label{\detokenize{test/ble_adv_scan_all:ble-5-0-ble-4-0}}
上面讲的只是蓝牙4.0的广播基本信息,实际上这里只讲了一小部分,还有很多内容未讲,实际上只是带大家入门,知道如何去看剩下的信息。ble5.0又引入了adv extend广播,就不止发31个字节。这些都是比较新的feature,通常安卓手机都是支持的。


\chapter{Nordic}
\label{\detokenize{nordic/nordic_nrf5x:nordic}}\label{\detokenize{nordic/nordic_nrf5x::doc}}

\section{1. introduce}
\label{\detokenize{nordic/nordic_nrf5x:introduce}}
hello


\section{2. introduce 2}
\label{\detokenize{nordic/nordic_nrf5x:introduce-2}}

\chapter{Indices and tables}
\label{\detokenize{index:indices-and-tables}}\begin{itemize}
\item {} 
\DUrole{xref,std,std-ref}{genindex}

\item {} 
\DUrole{xref,std,std-ref}{modindex}

\item {} 
\DUrole{xref,std,std-ref}{search}

\end{itemize}



\renewcommand{\indexname}{索引}
\printindex
\end{document}